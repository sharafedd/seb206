\documentclass{article}
\usepackage[utf8]{inputenc}
\usepackage[margin=1.25in]{geometry}
\usepackage{graphicx}
\usepackage{tabularx}
\usepackage{adjustbox} 
\usepackage{hyperref}
\usepackage{pdflscape} 
\usepackage{amsmath}
\usepackage{amsmath}


\begin{document}
	\begin{center}
    
        \includegraphics[width=0.3\textwidth]{images/uob_logo.png}\\[1em]
    
		\LARGE{\textbf{Final Year Project Proposal}} \\
        \vspace{1em}
        \Large{Echo: A Multimodal Spatio-Temporal Framework for Forecasting Dyadic Conflict and Cooperation} \\
        \vspace{1em}
        \normalsize\textbf{Sharaf-Eddine Boukhezer} \\
        \normalsize{seb206@student.bham.ac.uk} \\
        \vspace{1em}
        \normalsize{Supervisor: Dr. Leandro Minku} \\
        \vspace{1em}
        \normalsize{University of Birmingham} \\
        \normalsize{MSci Computer Science}
     
	\end{center}
    \begin{normalsize}
    
    	\section{Problem Statement}
        
        Forecasting geopolitical dynamics remains a challenging task due to cooperation and conflict between polities being affected by multiple factors, including historical patterns, structural relations, and sudden events and shocks. While media narratives provide useful signals, current methods are mostly unimodal, relying on either pure time series, text-based sentiment models, or static risk indices, which restricts their ability to adapt across different time horizons and evolving geopolitical contexts. \\

%--------------------------------------------------------------------------------

		\section{Related Work}
        
        Prior work in the field spans different methodological strategies: \\

        \begin{table}[H]
        \centering
        \renewcommand{\arraystretch}{0.9}
        \small
        \begin{tabularx}{\textwidth}{
            p{2.8cm}  % Paper
            p{2.6cm}  % Category
            X         % Model Input
            X         % Model Output
            p{2.2cm}  % Target Level
            p{1.8cm}  % Horizon
        }
        \toprule
        \textbf{Paper} & \textbf{Category} & \textbf{Model Input} & \textbf{Model Output} & \textbf{Target Level} & \textbf{Horizon} \\
        \midrule
        
        Chen et al. (2020) &
        Time-Series Dyad Model &
        GDELT dyad event counts (4 QuadClass types) + top-k related dyads &
        Next week's material conflict count &
        Single dyad &
        1-step (weekly) \\
        
        von der Maase (2025) -- HydraNet &
        Global Spatio-Temporal Forecasting &
        Past fatalities (priogrid-month) as spatiotemporal tensor &
        Conflict intensity 1--36 months ahead &
        Global grid (not dyads) &
        1--36m \\
        
        Croicu \& von der Maase (2025) &
        Text-Based Escalation Prediction &
        News text embeddings (Factiva) + actor metadata &
        Escalation/de-escalation classification &
        Dyads/actors &
        Next period \\
        
        Zakotianskyi (2025) &
        Feature-Rich Statistical / ML Models &
        100+ political/economic features from ViEWS \& UCDP &
        Conflict onset probability (1, 3, 6m) &
        All dyads/country pairs &
        Fixed horizons \\
        
        Liu \& Shen (2025) &
        Graph-Based ML / Cyber Relations &
        Graph of cyber relations + node features + threat-report text &
        Binary cyberattack prediction &
        All dyads (graph-wide) &
        Next period \\
        \bottomrule
        \end{tabularx}
        \caption{Overview of related forecasting approaches.}
        \label{tab:related_work}
        \end{table}
        
        Echo integrates elements from all of these, combining them into a single multimodal dyadic forecasting system. \\

%--------------------------------------------------------------------------------

	   	\section{Aim}
        
        Echo aims to build, evaluate, and analyse a multimodal forecasting system that predicts the future state of dyadic relations using four complementary signals:
        \begin{enumerate}
            \item Historical event patterns and trends
            \item Structural geopolitical relationships
            \item News-derived signals
            \item Country-level attributes (e.g., instability)
        \end{enumerate}
        
        The system will produce unified forward-looking forecasts of both event intensity (QuadClass counts) and relational polarity (Goldstein score) for any country pair at time t + n, where n denotes the forecasting horizon. \\

%--------------------------------------------------------------------------------

    	\section{Scope \& Boundaries}
        
        \subsection{In scope}
        \begin{itemize}
            \item Forecasting dyadic relations between countries in a global network
            \item Single-horizon prediction setup ($t \rightarrow t + n$)
            \item Two targets: Goldstein score and QuadClass event distribution
            \item Multimodal architecture combining time series, graph structure, text signals, and static features
            \item Baseline comparison with shared data splits
            \item Evaluation across increasing forecast horizons
            \item Ablation study
        \end{itemize}

        \subsection{Out of scope}
        \begin{itemize}
            \item Event-level prediction (Echo operates on aggregated time steps, not individual events).
            \item Forecasting specific event types or actor-level interactions beyond country dyads (focus limited to selected polities).
        \end{itemize}

%--------------------------------------------------------------------------------

    	\section{Methodology}

        \subsection{Data \& Preprocessing}

        \subsubsection*{Model Inputs and Data Sources}

        \begin{itemize}
            \item \textbf{Time-Series History (TS)}  
            dyad-level event aggregates (Goldstein, QuadClass counts, sentiment tone)  
            \textit{Source: GDELT 2.0 Global Events Database}
        
            \item \textbf{Text / News Signal (TX)}  
            BERT embeddings pooled per dyad-month from event descriptions/news text  
            \textit{Source: GDELT text fields + external news if needed}
        
            \item \textbf{Graph Structure (GR)}  
            Dynamic country graph based on alliances, trade, co-event frequency, and borders  
            \textit{Sources: COW, CEPII, ATOP, GDELT co-occurrence}
        
            \item \textbf{Static Dyad Features (ST)}  
            Geographic distance, GDP ratio, regime difference, shared organisations, and past hostility  
            \textit{Sources: CEPII, WGI, COW, World Bank}
        
        \end{itemize} \\

        \subsubsection*{Train/Validation/Test Split}
        All splits are chronological to prevent temporal leakage:
        \[
        \text{Train: } 2016\!-\!2010,\quad
        \text{Validation: } 2011,\quad
        \text{Test: } 2012\!-\!2013
        \]
        Each forecast horizon $t+n$ is trained and evaluated separately (e.g.\ $n=1,3,6,12$).
        
        \subsection{Baseline Models}

        \begin{enumerate}
            \item BERT-only MLP (text-only baseline)
            \item Na\"{i}ve last-value baseline
            \item LSTM (time-series only)
            \item Additional baseline models to be proposed
        \end{enumerate}
        
        We will also use the reported 70\% accuracy from Chen \textit{et al.} as a reference point for comparing Echo against existing research. \\
        
        \subsection{Proposed Model}

        \subsubsection*{Model Inputs (per dyad, per month)}
        \begin{itemize}
            \item \textbf{Time-Series Window} (TS): past $k$ months of Goldstein, QuadClass counts, tone, and rolling statistics
            \item \textbf{Textual Signal} (TX): pooled BERT embedding for the current dyad-month (BERT is frozen)
            \item \textbf{Graph Context} (GR): node embeddings from a GNN (GraphSAGE/GAT), combined into a dyad vector
            \item \textbf{Static Features} (ST): distance, GDP ratio, regime difference, trade links, alliance flags, etc.
        \end{itemize}

        \subsubsection*{Model Architecture}


        \subsection{Output Definition}

        For each dyad $(A,B)$ in the global country network at time $t$, the model produces a single-horizon forecast for time $t + n$ (in weeks). \\

        The prediction for a dyad is a 5-dimensional output vector:
        \[
        \hat{\mathbf{y}}_{A,B}(t+n) = 
        \begin{bmatrix}
        \text{Goldstein}_{t+n} \\[2pt]
        \text{Quad}_1\text{ (verbal cooperation)} \\[2pt]
        \text{Quad}_2\text{ (material cooperation)} \\[2pt]
        \text{Quad}_3\text{ (verbal conflict)} \\[2pt]
        \text{Quad}_4\text{ (material conflict)}
        \end{bmatrix}
        \] \\
        
        The model generates this vector \emph{simultaneously for all dyads}, resulting in a prediction matrix of size:
        \[
        \#\text{dyads} \;\times\; 5
        \] \\
        
        Note: the set of dyads is restricted to the 10 most active countries globally (ranked by event frequency). \\

        \subsection{Evaluation Protocol}

        The model will be evaluated along four dimensions:
        
        \begin{itemize}
            \item Across models (Echo vs.\ all baselines)
            \item Across time horizons (short-, medium-, and long-term; objective up to $t{+}36$ months)
            \item Across targets (Goldstein polarity vs.\ QuadClass event distribution)
            \item Across dyad types (e.g.\ allies, rivals, neutral pairs, high- vs.\ low-activity dyads)
        \end{itemize} \\

        Metrics:
        \begin{itemize}
            \item Goldstein $\rightarrow$ MAE, MSE, RMSE, $R^{2}$
            \item QuadClass $\rightarrow$ MSE, RMSE, Negative Binomial deviance, Distributional Calibration Error (proper scoring rule, distribution-level interpretation)
        \end{itemize}

%--------------------------------------------------------------------------------

    	\section{Work Plan}

        \begin{enumerate}
            \item \textbf{Build full data pipeline (ETL + feature store):} Build ETL, clean data, generate monthly dyad table.
            \item \textbf{Train baseline models:} Train all baseline models for $t{+}1$ horizon and for smaller data.
            \item \textbf{Implement Echo v0 (minimal multimodal version).}
            \item \textbf{Compare} Echo v0 vs.\ baselines on the same split and horizons.
            \item \textbf{Micro ablation} (only if Echo v0 $<$ best baseline). 
            \item \textbf{Full ablation study to find best Echo configuration:} Remove each modality, re-train, compare.
            \item \textbf{Final evaluation:} Best Echo vs.\ top 2--3 baselines on full test set and all horizons.
            \item \textbf{(Optional) Visualization Dashboard.}
            \item \textbf{(Optional) LLM interpretability:} Use an LLM to interpret and explain the predictions.
        \end{enumerate}

%--------------------------------------------------------------------------------

    	\section{Deliverables}
        
        \begin{itemize}
            \item \textbf{Full Data Pipeline}: Reproducible ETL process and dataset builder (train/val/test splits)
            \item \textbf{Final Model (Echo)}
            \item \textbf{Evaluation Report}: Baselines vs.\ Echo performance, accuracy per horizon and per metric, with full reproducibility details
            \item \textbf{Interactive Dashboard (Optional)}: Visualizes historical and predicted dyad trends (\url{https://echo-theta-coral.vercel.app/})
            \item \textbf{Code \& Documentation}: GitHub and GitLab repositories with technical documentation
            \item \textbf{Final Written Report}
        \end{itemize}

%--------------------------------------------------------------------------------

    	\section{Risks \& Mitigation}
        
        \begin{itemize}
            \item High computational cost: full-model training requires substantial GPU memory, especially with larger batch sizes
            \item Overfitting risk: a small number of high-interaction dyads may cause the model to memorise rather than generalise
            \item Data sparsity in low-activity dyads: many country pairs register near-zero events per month, destabilising training; smoothing or filtering to the most active dyads may be required
            \item Noisy or weakly relevant text signals may dilute predictive power
            \item Long-horizon performance degradation
            \item Concept drift: geopolitical relations shift through coups, wars, alliances, and sanctions, making historical patterns partially obsolete; restricting the input window may help
            \item Extreme shocks (e.g., invasions, regime collapses) disrupt historical trends and produce outlier targets
        \end{itemize}
        
%--------------------------------------------------------------------------------

        \section{References}

        \makeatletter
        \def\IEEEbibitemsep{0pt plus .5pt}
        \renewcommand{\section}[2]{}%
        \makeatother

        \subsection{Papers}
        \nocite{*}
        \bibliographystyle{IEEEtran}
        \bibliography{papers} \\
        
        \subsection{Datasets}
        \nocite{*}
        \bibliographystyle{IEEEtran}
        \bibliography{datasets}


\end{normalsize}
  
\end{document}
